% Witted proposal template, version 1.0
% 2022/05/25

\documentclass{article}
\usepackage[utf8]{inputenc}

\title{Project Title}
\author{Student name}
% Second Year of Bachelor in Physics 
% University of the Moon.
\date{Month, year}

\begin{document}

\maketitle

\section{Abstract}
Background or introduction  section provides  a description  of the basic  facts and importance  of the research area. Example of a possible abstract structure is: general field of work, open questions or gap (for you at least, or the team), your expected main work to address some of these, the expected significance of your work


% Example:
%Bio-acoustic is becoming one of the key approach to study underwater mammal populations. However at today still, there is not an open source documentation about the construction of an underwater acoustic sensors, limiting so its use only to fields where expensive commercial sensors could be purchased. With this project I would like so to understand, develop, test and open-source share an underwater acoustic sensor, which could be used for further research and development in other fields. Such project will enable future work on open source bio-acoustic.

\section{Problem statement and objectives }
Please describe the macro challenge of your project.

Also provides  a clear list and concise description  of the objectives (challenges) you need to address and the expected main outputs, as a report, chapter, code, data, prototype etc. If the project is a thesis, you can think the objectives can be chapters or main section of chapters, so the list of objectives will become a kind of draft outline. Having a clear list of outputs will help you focus and decide the most effective way to achieve it. Usually for 6 month intern or thesis project objectives are between 4 to 6 on average, but there is so rule.

% Example:
%The problem I need to solve is how to build and characterize an active echo sonar to measure the distance to the seafloor with a resolution of 1 cm and a maximum range of 150 meters in sweet and salty water as well.

%As a first objective, on the theory and modelling side, I need to understand and document the underwater acoustic physics and signal propagation from a source emitter, to a target and back to a receiver. The output will be a summary report on  the optimal frequencies and minimum impulse power to determine the distance within the chosen specs.  

%A second objective is to build a basic a piezoacoustic transducer to  experimentally establish the relationship between the acoustic and electric signals and identify the requirements for the driving and amplifying transducer electronics.  The output will be a build prototype of piezo transducer and a report presenting the setup and its first experimental characterization.

%...

\section{Required topics to review}
What is the necessary theoretical or background knowledge you need to comprehend to complete the project? What are your learning goals? declaring this help the supervision team assist you and guide you for how much as possible.

%example:
%For the theoretical understanding, I need to complete the read up and document the following topics:
%wave propagation in a medium [1], reflection and refraction of a wave in contact with the boundary of another medium [1],dissipation of power in an expanding spherical wave [2], detection theory [2], the active sonar equation [2], and the piezoelectric effect [3]. I have to understand as well the ambient acoustic noise and how it affects the signal and, so the measurement.
%This document will be the basis for the experimental realization of the sounder. 

\section{Validation approach}
 what is the validation approach you are going to use? For example, for software development could be unit testing or particular block development approach, for modelling could be comparison with other known models or with approximations at certain limits, for artificial intelligence could be ablation studies and comparison, while for experimental work could be system and unit testings, and comparison with expected models. 

\section{Moonshot (optional)}
Follow your intuition and learn something important in a quick but sometime "dirty"or potentially destructive way, it does not matter. Its purpose is to get out of safe step by step approach, and take giant leap of intuition and directly look for unknowns unknowns. 

\section{Your work legacy for the future world out there}
This section is related to how you need to setup your work final outputs such that they could become the stepping stones for others to practically build their own work on. For example, you can say that you will package your thesis code as shared libraries or frameworks with included developer documentation and examples. Or it would be a well prepared and documented dataset, or even a paper. For others we do not mean colleagues only but they can be also the general public. 

\section{Time plan and deliverable}
 Which is the estimated time plan for your project for each clear deliverable outputs? No thesis or inters has ever fully respect their time-plans, since there are always interesting o needed detours or stop changing them. But thinking about it, it is very useful to look into possible dependencies of work-to-do and also self asses if all goes to plan and how to correct the course to still achieve something.  No need to be super detailed, time span within 2-6 weeks is ok, include also breaks for exams or holidays if you think might impact your time-plan, so to self-assess the time you have available while you progress along the project.

% es: starting from 5 March 2022
% 5 -19 March 2022 -2 weeks for tech Report 1
% 20 March - 10 April 2022 3 week for dataset preparation
%

\section*{References (Examples)}

\begin{itemize}
    \item  Xavier Lurton (2010).  \textit{"An Introduction to Underwater Acoustics: Principles and Applications"}. Springer [1]

    \item  E.  J.  Tucholski (2006). \textit{" Underwater  Acoustics  and  Sonar  SP411  Handouts  and  Notes  Fall."} Physics  Department, U.  S.  Naval  Academy [2]
    \item John L. Butler, Charles H. Sherman (2016).\textit{Transducers and  Arrays  for Underwater Sound,"} 2nd Edition, Springer [3]
\end{itemize}


\end{document}

